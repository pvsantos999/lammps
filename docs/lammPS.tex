% ****** Start of file apssamp.tex ******
%
%   This file is part of the APS files in the REVTeX 4.2 distribution.
%   Version 4.2a of REVTeX, December 2014
%
%   Copyright (c) 2014 The American Physical Society.
%
%   See the REVTeX 4 README file for restrictions and more information.
%
% TeX'ing this file requires that you have AMS-LaTeX 2.0 installed
% as well as the rest of the prerequisites for REVTeX 4.2
%
% See the REVTeX 4 README file
% It also requires running BibTeX. The commands are as follows:
%
%  1)  latex apssamp.tex
%  2)  bibtex apssamp
%  3)  latex apssamp.tex
%  4)  latex apssamp.tex
%
% see http://web.mit.edu/8.13/www/revtex4-command-summary.pdf for a review of docclass options
\documentclass[%
% twocolumn,
preprint,
superscriptaddress,
%groupedaddress,
%unsortedaddress,
%runinaddress,
%frontmatterverbose, 
%preprint,
%preprintnumbers,
%nofootinbib,
%nobibnotes,
%bibnotes,
 amsmath,amssymb,
% prl,
 %aps,
%pra,
%prb,
%rmp,
%prstab,
%prstper,
prapplied,
%prx,
floatfix,
tightenlines,
]{revtex4-2}

\usepackage{graphicx}% Include figure files
\usepackage{dcolumn}% Align table columns on decimal point
\usepackage{bm}% bold math
\usepackage{hyperref}% add hypertext capabilities
%\usepackage[mathlines]{lineno}% Enable numbering of text and display math
%\linenumbers\relax % Commence numbering lines

\usepackage{tablefootnote}
\usepackage{threeparttable}

%\usepackage[showframe,%Uncomment any one of the following lines to test 
%%scale=0.7, marginratio={1:1, 2:3}, ignoreall,% default settings
%%text={7in,10in},centering,
%%margin=1.5in,
%%total={6.5in,8.75in}, top=1.2in, left=0.9in, includefoot,
%%height=10in,a5paper,hmargin={3cm,0.8in},
%]{geometry}


%%%%%%%%%%%%%%%%%%%%%%%%%%%%%%%%%%%%%%%%%%%%%%%%%
% for Supplementary material in revtex

%from https://support.authorea.com/en-us/article/how-to-create-an-appendix-section-or-supplementary-information-1g25i5a/
\newcommand{\beginsupplement}{%
        \setcounter{table}{0}
        \renewcommand{\thetable}{SM\arabic{table}}%
        \setcounter{figure}{0}
        \renewcommand{\thefigure}{SM\arabic{figure}}%
        \setcounter{equation}{0}
        \renewcommand{\theequation}{SM\arabic{equation}}%
        \setcounter{section}{0}
        \renewcommand{\thesection}{SM\arabic{section}}%
     }

  %\xdef\presupwidefigures{\arabic{figure*}}% save the current figure number
  %\xdef\presupfiguress{\arabic{figure*}}% save the current figure number
  %\renewcommand\thefigure*{S\fpeval{\arabic{figure}-\presupfiguress}}

%%%%%%%%%%%%%%%%%%%%%%%%%%%%%%%%%%%%%%%%%%%%%%%%%


%%%%%%%%%%%%%%%%%%%%%%%%%%%%%%%%%%%%%%%%%%%%%%%%%
% Revision tracking (change) package settings
% see: http://mirrors.ibiblio.org/CTAN/macros/latex/contrib/changes/changes.english.pdf

\usepackage{xcolor}
%\usepackage[draft]{changes}
\usepackage[final]{changes}
\definechangesauthor[color=green]{common}
\definechangesauthor[color=red, name={Paulo Santos}]{PVS}
%\definechangesauthor[color=blue]{MM}
\definechangesauthor[color=red]{MM}

%% Rather hacky definition of a plain remark/note by riding on \added
\setauthormarkup{}
%\let\newcite\cite 
%\newcommand{\mcite}[1]{\mbox{\cite{#1}}}
%\newcommand{\monlinecite}[1]{\mbox{\cite{#1}}}
\newcommand{\rPVS}[2]{\replaced[id=PVS]{#1}{#2}}
\newcommand{\aPVS}[1]{\added[id=PVS]{#1}}
\newcommand{\dPVS}[1]{\deleted[id=PVS]{#1}}
\newcommand{\cPVS}[1]{\comment[id=PVS]{#1}}
%%%%%%%%%%%%%%%%%%%%%%%%%%%%%%%%%%%%%%%%%%%%%%%%

%%%%%%%%%%%%%%%%%%%%%%%%%%%%%%%%%%%%%%%%%%%%%%%%
%%% PVS Main SAW parameters (definition.tex)
% SAW parameters
\newcommand{\lSAW}{{\lambda_{\mathrm{SAW}}}}		% acoustic wavelength
\newcommand{\vSAW}{{v_{\mathrm{SAW}}}}		% acoustic wavelength
\newcommand{\fSAW}{{f_{\mathrm{SAW}}}}		% acoustic wavelength

%%%% end %%%

\begin{document}

%\preprint{APS/123-QED}
\preprint{LAMMPS}

\title[LAMMPS]{Lammps - Molecular Dynamics Simulations }
% Force line breaks with \\


\author{Paulo V. Santos}
\email{santos@pdi-berlin.de}
 \affiliation{Paul-Drude-Institut f\"ur Festk\"orperelektronik, Leibniz-Institut im Forschungsverbund e.V., Berlin, Germany}
 

\date{\today}% It is always \today, today,
             %  but any date may be explicitly specified

\begin{abstract}
Working with lammps!
\end{abstract}

\maketitle
\section{Introduction}
\label{sec:intro}

\section{General info}

\begin{itemize}
\item Project website: 
\item Language reference: 
   \begin{itemize}
      \item \href{http://lammps.sandia.gov/doc}{http://lammps.sandia.gov/doc}
      \item github: \href{ https://github.com/lammps/lammps/}{https://github.com/lammps/lammps/}
   \end{itemize}
\item Tutorials
   \begin{itemize}
      \item Simple tutorial: \href{http://wp.df.uba.ar/gebi/wp-content/uploads/sites/9/2016/06/lammps.pdf}{lammps for dummies}
      \item Tutorial: \href{https://icme.hpc.msstate.edu/mediawiki/index.php/LAMMPS_Tutorial_1}{tutorial 1}%{https://icme.hpc.msstate.edu/mediawiki/index.php/LAMMPS\_Tutorial\_1}
      %\item Tutorial: \href{https://icme.hpc.msstate.edu/mediawiki/index.php/LAMMPS_Tutorial_1}{tutorial 1 https://icme.hpc.msstate.edu/mediawiki/index.php/LAMMPS-Tutorial-1}
      \item Examples: \href{https://icme.hpc.msstate.edu/mediawiki/index.php/LAMMPS_tutorials}{tutorial 2}%{https://icme.hpc.msstate.edu/mediawiki/index.php/LAMMPS\_tutorials}
      \item \href{http://puccini.che.pitt.edu/~karlj/Classes/CHE3935/MD_simulations.pdf}{Good for input file: slide 40 on!}
   \end{itemize}

%{https://icme.hpc.msstate.edu/mediawiki/index.php/LAMMPS\_Tutorial\_1}
\item making videos: see linux-docs.pdf
\end{itemize}

\noindent{\em Acknowledgements: }{We thank $\dots$. PVS acknowledge financial support from the German DAAD (grant 57314018) and DFG (grant 4056192179), respectively.}


%%%%%%%%%%%%%%%%%%%%%%%
%\bibliographystyle{apsrev4-2}
%\bibliographystyle{plain}
\IfFileExists{x:/sawoptik_databases/jabref/literature.bib}
{   \def\litdir{x:/sawoptik_databases/jabref} }
{	\def\litdir{c:/myfiles/jabref} }

%\bibliography{\litdir/literature,\litdir/mypapers,GaussianFocus}
\bibliography{\litdir/literature,\litdir/mypapers}
\end{document}
%%%%%%%%%%%%%%%%%%%%%%%

